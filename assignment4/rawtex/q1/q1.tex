\documentclass[12pt, a4paper]{article}
\usepackage[margin = 1in, top=1.3in]{geometry}
\usepackage[english]{babel}
\usepackage[utf8]{inputenc}
\usepackage{fancyhdr}
\usepackage{amsmath}
\usepackage{bm}
\usepackage{graphicx}
\usepackage{subcaption}
\usepackage[font=small,labelfont=bf]{caption}
 
\pagestyle{fancy}
\fancyhf{}
\rhead{\small{Shaan Ul Haque(180070053)\\ Samarth Singh (180050090) \\ Niraj Mahajan (180050069)}}
\lhead{CS-663 Assignment-3 : Question 1}
\rfoot{Page 1.\thepage}
 
\begin{document}
\vspace*{-22pt}
\section{Code}
Below is the MATLAB routine function for SVD of a matrix.
\begin{verbatim}
function [U S V]=MySVD(A)
    [U eig_values1]=eig(A*A');
    [V eig_values2]=eig(A'*A);
    [m m]=size(U);
    [n n]=size(V);
    if m>n
        S=eig_values1(:,m-n+1:end);
    else
        S=eig_values2(n-m+1:end,:);
    end
    S=S.^(0.5);
    LHS=A*V;
    RHS=U*S;
    % To check for the sign of eigenvectors because with the constraint
    % a'a=1 two vectors a and -a are possible
    for i=1:n
        if norm(LHS(:,i)-RHS(:,i))>=10e-2
            U(:,i)=-1*U(:,i);
        end
end
\end{verbatim}
\section{Example}
Let us generate a random $5\times3$ matrix using randi function and find its SVD and verify the theorem.
\subsection{Parameters}
\begin{itemize}
    \item Seed = 0
    \item Range of random numbers = [0,10]
    \item Size = $5\times3$
\end{itemize}
A=
\begin{bmatrix}
     8   &  1  &   1\\
     9  &   3 &   10\\
     1  &   6 &   10\\
    10  &  10 &    5\\
     6  &  10 &    8\\
\end{bmatrix}
\subsection{Output}
U=
\begin{bmatrix}
     0.1894 & -0.7053  & -0.2011  & -0.6151 & 0.2187\\
   -0.0020  &  0.4422  & -0.7470  & -0.0899 &  0.4882\\
   -0.3272  & -0.5211 &  -0.1428  &  0.6775 & 0.3770\\
   -0.5596 &   0.1702 &   0.4984  & -0.3370 & 0.5440\\
    0.7375  &  0.0803  &  0.3644  &  0.2026 & 0.5251\\
\end{bmatrix}

S=
\begin{bmatrix}
     0    &     0     &    0\\
         0    &     0   &      0\\
    6.6128    &     0   &      0\\
         0  &  9.0725   &      0\\
         0     &    0  & 26.3051\\
\end{bmatrix}

V=
\begin{bmatrix}
     -0.1972 &  -0.7944  &  0.5745\\
    0.8059  &  0.2023  &  0.5564\\
   -0.5583  &  0.5727  &  0.6003\\
\end{bmatrix}

Computing $U*S*V^{T}$ gives us back A. Hence, SVD stands verified. For better scrutiny, we have uploaded the MATLAB code as well.
\end{document}

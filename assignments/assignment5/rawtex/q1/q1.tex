\documentclass[12pt, a4paper]{article}
\usepackage[margin = 1in, top=1.3in]{geometry}
\usepackage[english]{babel}
\usepackage[utf8]{inputenc}
\usepackage{fancyhdr}
\usepackage[fleqn]{amsmath}
\usepackage{mathtools}
\usepackage{tabto}
\usepackage{bm}
\usepackage{graphicx}
\graphicspath{{./images/}}
\usepackage[font=small,labelfont=bf]{caption}
 
\pagestyle{fancy}
\fancyhf{}
\rhead{\small{Shaan Ul Haque(180070053)\\ Samarth Singh (180050090) \\ Niraj Mahajan (180050069)}}
\lhead{CS-663 Assignment-5 : Question 1}
\rfoot{Page 1.\thepage}
 
\begin{document}
\vspace*{-22pt}
\section*{Question 1}
\quad Given the image taken when the outside scene is in focus $g_1$, and the image when the reflection off the window is in focus $g_2$, along with the respective blurring kernels $h_1$ and $h_2$, we have \\
$$g_1 = f_1 + h_2*f_2$$
$$g_2 = f_2 + h_1*f_1$$
By applying Fourier transform on these equations, we get,
$$G_1 = F_1 + H_2F_2$$
$$G_2 = F_2 + H_1F_1$$
On further solving the system of linear equations for $F_1$, $F_2$,
$$F_1 = \frac{G_1 - H_2G_2}{1-H_1H_2}$$
$$F_2 = \frac{G_2 - H_1G_1}{1-H_1H_2}$$
Now, to obtain $f_1$ and $f_2$, we simply take the Fourier Inverse Transform of $F_1$ and $F_2$
$$\boxed{f_1 = \mathcal{F}^{-1}(F_1)}$$
$$\boxed{f_2 = \mathcal{F}^{-1}(F_2)}$$

\subsection*{Problem in the solution obtained}
The inherent problem in the formula derived is that the Fourier transforms of the blurring kernels act as low pass filters. So, for low frequencies, $H_1H_2$ will approach 1. If we look at the denominator of the formulae $1 - H_1H_2$, we notice that for low frequencies, the denominator tends to $\infty$. This is will amplify the noise around low frequencies and hence is not ideal.
\end{document}
